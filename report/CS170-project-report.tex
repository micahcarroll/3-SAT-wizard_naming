%%%%%%%%%%%%%  DOCUMENT SETUP %%%%%%%%%%%%%%%%%   
\documentclass{article}
\renewcommand{\baselinestretch}{1.05}
\usepackage{amsmath,amsthm,verbatim,amssymb,amsfonts,amscd, graphicx}
\usepackage{graphics}
\usepackage{fontspec}
\usepackage{listings}
\setmainfont{Calluna}
\setmonofont[Scale=0.8]{Menlo}
\lstset{basicstyle=\ttfamily,breaklines=true}
\textheight23.0cm
\textwidth16.5cm
\footskip1.0cm

%%%%%%%%%%%%%  THEOREMS  %%%%%%%%%%%%%%%%%
\theoremstyle{plain}
  \newtheorem{theorem}{Theorem}[section]
  \newtheorem{conjecture}[theorem]{Conjecture}
  \newtheorem{problem}[theorem]{Problem}
  \newtheorem{assumption}[theorem]{Assumption}
  \newtheorem{heuristic}[theorem]{Heuristic}
  \newtheorem{proposition}[theorem]{Proposition}
  \newtheorem{fact}[theorem]{Fact}
  \newtheorem{lemma}[theorem]{Lemma}
  \newtheorem{corollary}[theorem]{Corollary}
  \newtheorem{claim}[theorem]{Claim}
  \newtheorem{exercise}[theorem]{Exercise}
  \numberwithin{theorem}{subsection}
\theoremstyle{definition}
  \newtheorem{definition}[theorem]{Definition}
  \newtheorem{example}[theorem]{Example}
  \newtheorem{remark}[theorem]{Remark}
  \numberwithin{equation}{subsection}

%%%%%%%%%%%%% SHORTCUTS %%%%%%%%%%%%%%%%%%%%
\newcommand\g{{\mathbb{g}}}
\newcommand\R{{\mathbb{R}}}
\newcommand\C{{\mathbb{C}}}
\newcommand\Z{{\mathbf{Z}}}
\newcommand\D{{\mathbf{D}}}
\newcommand\Q{{\mathbf{Q}}}
\newcommand\G{{\mathbf{G}}}
\newcommand\I{{\mathbf{I}}}
\renewcommand\P{{\mathbf{P}}}
\newcommand\E{{\mathbf{E}}}
\newcommand\U{{\mathbf{U}}}
\newcommand\A{{\mathbf{A}}}
\newcommand\Var{\mathbf{Var}}
\renewcommand\Im{{\operatorname{Im}}}
\renewcommand\Re{{\operatorname{Re}}}
\newcommand\eps{{\varepsilon}}
\newcommand\trace{\operatorname{trace}}
\newcommand\supp{\operatorname{supp}}
\newcommand\tr{\operatorname{tr}}
\newcommand\dist{\operatorname{dist}}
\newcommand\Span{\operatorname{Span}}
\newcommand\sgn{\operatorname{sgn}}
\newcommand\dive{\operatorname{div}}
% \renewcommand\a{x}
% \renewcommand\b{y}

% \newcommand\bm{{\mathbf{m}}}
\newcommand\dd{\partial}
\newcommand\smd{\hbox{smd}}
\newcommand\Ba{{\mathbf a}}
\newcommand\Bb{{\mathbf b}}
\newcommand\Bc{{\mathbf c}}
\newcommand\Bd{{\mathbf d}}
\newcommand\Be{{\mathbf e}}
\newcommand\Bf{{\mathbf f}}
\newcommand\Bg{{\mathbf g}}
\newcommand\Bh{{\mathbf h}}
\newcommand\Bi{{\mathbf i}}
\newcommand\Bj{{\mathbf j}}
\newcommand\Bk{{\mathbf k}}
\newcommand\Bl{{\mathbf l}}
\newcommand\Bm{{\mathbf m}}
\newcommand\Bn{{\mathbf n}}
\newcommand\Bo{{\mathbf o}}
\newcommand\Bp{{\mathbf p}}
\newcommand\Bq{{\mathbf q}}
\newcommand\Bs{{\mathbf s}}
\newcommand\Bt{{\mathbf t}}
\newcommand\Bu{{\mathbf u}}
\newcommand\Bv{{\mathbf v}}
\newcommand\Bw{{\mathbf w}}
\newcommand\Bx{{\mathbf x}}
\newcommand\By{{\mathbf y}}
\newcommand\Bz{{\mathbf z}}
\newcommand\p{{\mathbf p}}
\newcommand\bi{{\mathbf i}}
%

\newcommand\BA{{\mathbf A}}
\newcommand\BB{{\mathbf B}}
\newcommand\BC{{\mathbf C}}
\newcommand\BD{{\mathbf D}}
\newcommand\BE{{\mathbf E}}
\newcommand\BF{{\mathbf F}}
\newcommand\BG{{\mathbf G}}
\newcommand\BH{{\mathbf H}}
\newcommand\BI{{\mathbf I}}
\newcommand\BJ{{\mathbf J}}
\newcommand\BK{{\mathbf K}}
\newcommand\BL{{\mathbf L}}
\newcommand\BM{{\mathbf M}}
\newcommand\BN{{\mathbf N}}
\newcommand\BO{{\mathbf O}}
\newcommand\BP{{\mathbf P}}
\newcommand\BQ{{\mathbf Q}}
\newcommand\BS{{\mathbf S}}
\newcommand\BT{{\mathbf T}}
\newcommand\BU{{\mathbf U}}
\newcommand\BV{{\mathbf V}}
\newcommand\BW{{\mathbf W}}
\newcommand\BX{{\mathbf X}}
\newcommand\BY{{\mathbf Y}}
\newcommand\BZ{{\mathbf Z}}
\renewcommand\Pr{{\mathbf P }}

%cal letter

\newcommand\CA{{\mathcal A}}
\newcommand\CB{{\mathcal B}}
\newcommand\CC{{\mathcal C}}
\renewcommand\CD{{\mathcal D}}
\newcommand\CE{{\mathcal E}}
\newcommand\CF{{\mathcal F}}
\newcommand\CG{{\mathcal G}}
\newcommand\CH{{\mathcal H}}
\newcommand\CI{{\mathcal I}}
\newcommand\CJ{{\mathcal J}}
\newcommand\CK{{\mathcal K}}
\newcommand\CL{{\mathcal L}}
\newcommand\CM{{\mathcal M}}
\newcommand\CN{{\mathcal N}}
\newcommand\CO{{\mathcal O}}
\newcommand\CP{{\mathcal P}}
\newcommand\CQ{{\mathcal Q}}
\newcommand\CS{{\mathcal S}}
\newcommand\CT{{\mathcal T}}
\newcommand\CU{{\mathcal U}}
\newcommand\CV{{\mathcal V}}
\newcommand\CW{{\mathcal W}}
\newcommand\CX{{\mathcal X}}
\newcommand\CY{{\mathcal Y}}
\newcommand\CZ{{\mathcal Z}}

%number theory
\newcommand\condo{{\bf C0}}
\newcommand\condone{{\bf C1}}
\newcommand\N{{\mathbb N}}
\newcommand\BBQ {{\mathbb Q}}
\newcommand\BBI {{\mathbb I}}
\newcommand\BBR {{\mathbb R}}
\newcommand\BBZ {{\mathbb Z}}
\newcommand\BBC{{\mathbb C}}

% tilde
\newcommand\ta{{\tilde a}}
\newcommand\tb{{\tilde b}}
\newcommand\tc{{\tilde c}}
\newcommand\td{{\tilde d}}
\newcommand\te{{\tilde e}}
\newcommand\tf{{\tilde f}}
\newcommand\tg{{\tilde g}}
% \newcommand\th{{\tilde h}}
\newcommand\ti{{\tilde i}}
\newcommand\tj{{\tilde j}}
\newcommand\tk{{\tilde k}}
\newcommand\tl{{\tilde l}}
\newcommand\tm{{\tilde m}}
\newcommand\tn{{\tilde n}}
%\newcommand\to{{\tilde o}}
\newcommand\tp{{\tilde p}}
\newcommand\tq{{\tilde q}}
\newcommand\ts{{\tilde s}}
%\newcommand\tt{{\tilde t}}
\newcommand\tu{{\tilde u}}
\newcommand\tv{{\tilde v}}
\newcommand\tw{{\tilde w}}
\newcommand\tx{{\tilde x}}
\newcommand\ty{{\tilde y}}
\newcommand\tz{{\tilde z}}
\newcommand\ep{{\epsilon}}
%\newcommand\trace{{\operatorname{trace}}}
\newcommand\sign{{\operatorname{sign}}}
%\newcommand\hess{{\operatorname{Hess}}}
\newcommand\Dyson{{\operatorname{Dyson}}}
\renewcommand\th{{\operatorname{th}}}

\def\x{{\bf X}}
\def\y{{\bf Y}}
%\def\I#1{{\bf 1}_{#1}}
\def\mb{\mbox}

\newcommand\eqDist{{\overset{d}{=}}}
%%%%%%%%%%%%%%  PAGE SETUP %%%%%%%%%%%%%%%%%
\parindent = 5 pt
% \parskip = 12 pt
\usepackage[margin=1.5in]{geometry}
\usepackage{fancyhdr}
\pagestyle{headings}
\usepackage{fancyvrb}
\usepackage{hyperref}
\usepackage{xcolor}
\usepackage{mdframed}
\hypersetup{
    colorlinks,
    citecolor=black,
    filecolor=black,
    linkcolor=black,
    urlcolor=black
}
% \lhead{Kenneth Lai} % Left Header
% \rhead{\thepage} % Right Header
\cfoot{} % Center Foot (empty)

%%%%%%%%%%%%%%  Title and TOC %%%%%%%%%%%%%%%%%
\begin{document}
\title{CS170 Project Report (Fall 2017)}
\author{Micah Carroll, Orkun Duman, Kenneth Lai}
\maketitle
\tableofcontents
\eject
%%%%%%%%%%%%%%  START CONTENT %%%%%%%%%%%%%%%%%
\section{Main Idea.} We propose a reduction of the problem into a 3-Satisfiability(3-SAT) and solved it using simulated annealing. Suppose that we have $n$ wizards in the input files with random values denoted $w_1, \dots, w_n$. Then we produce ${n \choose 2}$ literals $X_{i, j}$ where \[X_{i, j} \Rightarrow w_i < w_j\] \[\neg X_{i, j} \Rightarrow w_i > w_j\] Then given the input files of relative ordering of wizards, we can generate our satisfiability clauses. In particular, each line $w_i, w_j, w_k$ can be translated as $(w_k < w_i \vee w_k > w_j) \wedge (w_k < w_j \vee w_k > w_i)$. Equivalently, \[(X_{k, i} \vee X_{j, k}) \wedge (X_{k, j} \vee X_{i, k})\] Where we naturally define $X_{i, j} = \neg X_{j, i}$ if one of them is not defined. Also, we add extra condition of transitivity for comparison, since $w_i < w_j \wedge w_j < w_k \Rightarrow w_i < w_k$ and $w_i > w_j \wedge w_j > w_k \Rightarrow w_i > w_k$. In our literals, this means we have a clauses \[(\neg X_{i, j} \vee \neg X_{j, k} \vee X_{i, k}) \wedge (X_{i, j} \vee  X_{j, k} \vee \neg X_{i, k})\]for each triplet. We aggregate all the clause mentioned above to define a 3SAT problem. For the smaller inputs (up to around 100) we simply solve it with a python inplemented 3SAT library (pycosat). For the inputs with larger input, we run a local search with simulated annealing: assigning literals and optimize over the number of constraints violated. With well chosen parameters we reached global optimum for all our problems. Then we use assignments of $X_{i, j}$ that satisfy all our constraints to recreate a valid ordering of the wizards.

\section{Libraries.}
\subsection{pycosat} We used an python implemented 3SAT solver initially to solve inputs with sizes up to $~100$. To install, run \texttt{pip install pycosat} on the terminal. 
\subsection{simanneal} \texttt{simanneal} is a multipupose simmulated annealing library. We used it to do local search to solve large input sizes. To install, run \texttt{pip install simanneal} on the terminal.

\section{Instructions.}
Run the solver by:\\
\texttt{python3 solver.py input.in output.out} \\\\
If you want to print verbose debug messages add the \texttt{ -d } flag: \\
\texttt{python3 solver.py input.in output.out -d} \\\\
You can also solve all files in a directory in order and pipe the outputs to another directory:\\
\texttt{python3 solver.py phase2/inputs phase2/outputs}\\

\section{Code.}
The files we wrote include: 
\begin{enumerate}
    \item \texttt{SAT3.py}
    \item \texttt{solver.py}
    \item \texttt{code\_generator.py} (used to generate random test inputs)
    \item \texttt{randomized3SAT.py} (deprecated)
    \item \texttt{solver2.py} (deprecated)
\end{enumerate}

\subsection{SAT3.py}
includes class files for solvers
\begin{lstlisting}
    import pycosat
    
    class SAT3(object):
        def __init__(self, clauses, variables):
            self.clauses = clauses
            self.variables = variables
    
    class LocalSearch(SAT3):
        def solve(self, depth=1000):
            for i in range(depth):
                self.switch_one()
            
        def switch_one(self):
            for clause in self.clauses: 
                if not clause.satisfied:
                    best_change = 0
                    best_variable = None
                    for var in clause.variables:
                        change = self.try_switch(var)
                        print(change)
                        if change > best_change:
                            best_change = change
                            best_variable = var
    
                    if best_variable:
                        print("Switching {} with {} change".format(best_variable, best_change))
                        best_variable.switch()
                        break
    
        def try_switch(self, variable):
            current = self.num_satisfied()
            variable.switch()
            with_switch = self.num_satisfied()
            variable.switch()
            return with_switch - current
    
        def num_satisfied(self):
            return sum([clause.satisfied for clause in self.clauses])
    
    class PycosatSolver(SAT3):
    
        def solve(self):
            pycosat_clauses = self.convert_to_pycosat(self.variables.encoder)
            pycosat_solution = pycosat.solve(pycosat_clauses)
            self.set_variables_from_pycosat(pycosat_solution, self.variables.decoder)
    
        def convert_to_pycosat(self, encoder):
            pycosat_instance = []
            for clause in self.clauses:
                pycosat_clause = []
                for item in zip(clause.variables, clause.truth_values):
                    truth_value = item[1]
                    encoded_variable = encoder[item[0]]
                    if not truth_value:
                        encoded_variable *= -1
                    pycosat_clause.append(encoded_variable)
                pycosat_instance.append(pycosat_clause)
            return pycosat_instance
            
        def set_variables_from_pycosat(self, pycosat_solution, decoder):
            for var in pycosat_solution:
                if var < 0:
                    decoder[-var].state = 0
                else:
                    decoder[var].state = 1
\end{lstlisting}

\subsection{solver.py}
Include various structs and the main solver method
\begin{lstlisting}
    import os
    import re
    import time
    import math
    import random
    import pycosat
    import argparse
    import itertools
    import simanneal
    import output_validator
    
    DEBUG = False
    
    class Variable:
        def __init__(self, wizard1, wizard2):
            self.wizard1 = min(wizard1, wizard2)
            self.wizard2 = max(wizard1, wizard2)
        
        def __eq__(self, other):
            return self.wizard1 == other.wizard1 and self.wizard2 == other.wizard2
    
        def __hash__(self):
            return hash(self.wizard1 + self.wizard2)
        
        def __repr__(self):
            return "Variable: {} BEFORE {}".format(self.wizard1, self.wizard2)
    
    class VariableList:
        def __init__(self, wizards):
            combinations = list((itertools.combinations(wizards, 2)))
            variables = [Variable(wizard1, wizard2) for wizard1, wizard2 in combinations]
            self.encoder = {}
            for i in range(len(variables)):
                self.encoder[variables[i]] = i + 1
            
        def encode_variable(self, variable):
            return self.encoder[variable]
        
        def __len__(self):
            return len(self.encoder)
    
        def __repr__(self):
            result = "Variable List <size: {}>".format(len(self.encoder))
            for variable in self.encoder:
                result += "\n{} -> {}".format(self.encoder[variable], str(variable))
            return result
    
    class Literal:
        def __init__(self, variable, value):
            self.variable = variable
            self.value = value
        
        def to_pycosat(self, variable_list):
            multiplier = 1 if self.value else -1
            return variable_list.encode_variable(self.variable) * multiplier
        
        def __repr__(self):
            assigned = "TRUE" if self.value else "FALSE"
            return "Literal: <{}> set to {}".format(str(self.variable), assigned)
    
    class Constraint:
        def __init__(self, triplet):
            assert(len(triplet) == 3)
            self.bound1 = triplet[0]
            self.bound2 = triplet[1]
            self.middle = triplet[2]
        
        # For each constraint "middle not between bound1 and bound2"
        # return a caluse of the form:
        # (bound1 < middle OR bound2 > middle) AND (bound2 < middle OR bound1 > middle).
        def to_clause(self, variable_list):
            a1 = Literal(Variable(self.bound1, self.middle), self.bound1 < self.middle)
            a2 = Literal(Variable(self.bound2, self.middle), self.bound2 > self.middle)
            b1 = Literal(Variable(self.bound2, self.middle), self.bound2 < self.middle)
            b2 = Literal(Variable(self.bound1, self.middle), self.bound1 > self.middle)
            return [Clause([a1, a2]), Clause([b1, b2])]
    
    class Clause(object):
        def __init__(self, literals):
            self.literals = literals
        
        def to_pycosat(self, variable_list):
            return [literal.to_pycosat(variable_list) for literal in self.literals]
    
        def __repr__(self):
            return "Clause:" + " OR ".join([str(literal) for literal in self.literals])
    
    def solve(num_wizards, num_constraints, wizards, constraints, data = None):
        # Pre-processing.
        processing_start = time.time()
        if DEBUG:
            print("Input with {} wizards and {} constraints.".format(len(wizards), len(constraints)))
        random.shuffle(wizards)
    
        # Generate all possible variables of 2 wizards each.
        variables = VariableList(wizards)
        assert(len(variables) == len(wizards) * (len(wizards) - 1) / 2)
        if DEBUG:
            print(variables)
    
        # Convert constraints to SAT clauses.
        clauses = []
        for triplet in constraints:
            constraint = Constraint(triplet)
            clause_pair = constraint.to_clause(variables)
            clauses.extend(clause_pair)
        assert(len(clauses) == len(constraints) * 2)
    
        # Add 3-Term clauses to prevent loops.
        sorted_wizards = sorted(wizards)
        for i in range(len(sorted_wizards)):
            for j in range(i + 1, len(sorted_wizards)):
                for k in range(j + 1, len(sorted_wizards)):
                    w1, w2, w3 = sorted_wizards[i], sorted_wizards[j], sorted_wizards[k]
                    a1 = Literal(Variable(w1, w2), False)
                    a2 = Literal(Variable(w1, w3), True)
                    a3 = Literal(Variable(w2, w3), False)
                    clauses.append(Clause([a1, a2, a3]))
                    b1 = Literal(Variable(w1, w2), True)
                    b2 = Literal(Variable(w1, w3), False)
                    b3 = Literal(Variable(w2, w3), True)
                    clauses.append(Clause([b1, b2, b3]))
        assert(len(clauses) == len(constraints) * 2 + len(list(itertools.combinations(range(len(wizards)), 3))) * 2)
        if DEBUG:
            print("Clauses (first 100):\n", clauses[:100])
    
        # Solve using pycosat.
        pycosat_input = [clause.to_pycosat(variables) for clause in clauses]
        algorithm_start = time.time()
        print("Calling Pycosat to solve the problem.")
        pycosat_output = pycosat.solve(pycosat_input)
        print("Pycosat returned an assignment.")
        algorithm_duration = round(time.time() - algorithm_start, 2)
    
        # Decode pycosat solution.
        assignments = {}
        for assignment in pycosat_output:
            assignments[abs(assignment)] = True if assignment >= 0 else False
        if DEBUG:
            print("Pycosat Assignments:\n", pycosat_output)
    
        # Find a valid ordering of wizards.
        solution = []
        for wizard in wizards:
            target_index = 0
            for i in range(len(solution)):
                variable = Variable(solution[i], wizard)
                value = assignments[variables.encode_variable(variable)]
                if variable.wizard1 != solution[i]:
                    value = not value
                if value:
                    target_index = i + 1
            solution.insert(target_index, wizard)
        if DEBUG:
            print("Solution:", solution)
        
        # Completion info.
        processing_duration = round(time.time() - processing_start, 2) - algorithm_duration
        print("Solver complete. Algorithm took {} seconds. Processing took {} seconds.".format(algorithm_duration, processing_duration))
        return solution
    
    class WizardSolver(simanneal.Annealer):
        Tmax = 60       # Max (starting) temperature (over-written below)
        Tmin = 0.01        # Min (ending) temperature
        steps = 1000000  # Number of iterations
        updates = steps / 100   # Number of updates (by default an update prints to stdout)
    
        def __init__(self, wizards, constraints):
            self.constraints = constraints
            #self.Tmax = 10 + math.sqrt(len(constraints))
            super(WizardSolver, self).__init__(wizards) 
    
        def move(self):
            a = random.randint(0, len(self.state) - 1)
            b = random.randint(0, len(self.state) - 1)
            self.state[a], self.state[b] = self.state[b], self.state[a]
        
        def energy(self):
            output_ordering_set = set(self.state)
            output_ordering_map = {k: v for v, k in enumerate(self.state)}
            not_satisfied = 0
            for constraint in self.constraints:
                c = constraint # Creating an alias for easy reference
                m = output_ordering_map # Creating an alias for easy reference
                wiz_a = m[c[0]]
                wiz_b = m[c[1]]
                wiz_mid = m[c[2]]
                if (wiz_a < wiz_mid < wiz_b) or (wiz_b < wiz_mid < wiz_a):
                    not_satisfied += 1
            return not_satisfied
    
    def anneal(num_wizards, num_constraints, wizards, constraints, data = None):
        # Pre-processing.
        algorithm_start = time.time()
        if DEBUG:
            print("Input with {} wizards and {} constraints.".format(len(wizards), len(constraints)))
        random.shuffle(wizards)
    
        # Start simulated annealing.
        start_state = data if data is not None else wizards
        solver = WizardSolver(start_state, constraints)
        print("Starting with ordering where {} constrains are violated.".format(solver.energy()))
        if solver.energy() < 100:
            solver.Tmax = 1
        if solver.energy() < 25:
            solver.Tmax = 0.5
        if solver.energy() < 10:
            solver.Tmax = 0.01
        solution, num_constraints_failed = solver.anneal()
        
        # Completion info.
        algorithm_duration = round(time.time() - algorithm_start, 2)
        print("\nSolver complete. Algorithm took {} seconds.".format(algorithm_duration))
        return solution
    
    """
    ======================================================================
       Input parsing happens below this line.
    ======================================================================
    """
    
    def read_input(filename):
        with open(filename) as f:
            num_wizards = int(f.readline())
            num_constraints = int(f.readline())
            constraints = []
            wizards = set()
            for _ in range(num_constraints):
                c = f.readline().split()
                constraints.append(c)
                for w in c:
                    wizards.add(w)
        wizards = list(wizards)
        return num_wizards, num_constraints, wizards, constraints
    
    def write_output(filename, solution):
        with open(filename, "w") as f:
            for wizard in solution:
                f.write("{0} ".format(wizard))
    
    def atoi(text):
        return int(text) if text.isdigit() else text
    
    def natural_keys(text):
        '''
        alist.sort(key=natural_keys) sorts in human order
        http://nedbatchelder.com/blog/200712/human_sorting.html
        (See Toothy's implementation in the comments)
        '''
        return [atoi(c) for c in re.split('(\d+)', text)]
    
    if __name__ == "__main__":
        parser = argparse.ArgumentParser(description = "Constraint Solver.")
        parser.add_argument(
            "input",
            type=str,
            help = "provide a file/folder where inputs should be read from")
        parser.add_argument(
            "output",
            type=str,
            help = "provide a file/folder where outputs should be saved")
        parser.add_argument(
            "--anneal", "-a",
            dest="anneal",
            action="store_true",
            help="use simulated annealing instead of 3SAT")
        parser.add_argument(
            "--start",
            dest="start",
            help="use the wizard ordering in this file as the starting state")
        parser.add_argument(
            "--debug", "-d",
            dest="debug",
            action="store_true",
            help="print out verbose debug messages")
        args = parser.parse_args()
    
        if args.debug:
            DEBUG = True
    
        start_state = None
        if args.start:
            with open(args.start, "r") as start_file:
                start_state = start_file.readline().split()
        
        inputs = [args.input]
        if os.path.isdir(args.input):
            inputs = [os.path.join(args.input, f) for f in os.listdir(args.input) if os.path.isfile(os.path.join(args.input, f)) and f.endswith(".in")]
            inputs.sort(key=natural_keys)
        
        for input_file in inputs:
            if not os.path.isdir(args.output):
                output_file = args.output
            else:
                output_file = os.path.join(args.output, os.path.split(input_file)[1].replace(".in", ".out").replace("input", "output"))
            if len(inputs) > 1 and os.path.isfile(output_file):
                try:
                    if len(output_validator.processInput(input_file, output_file)[2]) == 0:
                        print("File {} already has a valid solution in {}, skipping.".format(input_file, output_file))
                        continue
                except:
                    pass
            num_wizards, num_constraints, wizards, constraints = read_input(input_file)
            print("Solving file: {} ({} wizards, {} constraints)".format(input_file, num_wizards, num_constraints))
            f = anneal if args.anneal else solve
            solution = f(num_wizards, num_constraints, wizards, constraints, start_state)
            write_output(output_file, solution)
            constraints_satisfied, num_constraints, constraints_failed = output_validator.processInput(input_file, output_file)
            if constraints_satisfied == num_constraints:
                print("Solution file {} verified!".format(output_file))
            else:
                print("Solution file {} did not satisfy {}/{} constraints.".format(output_file, len(constraints_failed), num_constraints))
    
\end{lstlisting}

%%%%%%%%%%%%%%  END CONTENT %%%%%%%%%%%%%%%%%
\end{document}